\documentclass[12pt]{article}
%\usepackage{fontspec}
\usepackage{amsmath}
\usepackage{amsfonts}
\usepackage{graphicx}
\usepackage{enumitem}
\usepackage{amssymb}
\usepackage{relsize}
\usepackage{ upgreek }
\usepackage[utf8]{inputenc}
\usepackage[russian]{babel}
\usepackage[left=2cm,right=2cm,
top=2cm,bottom=2cm,bindingoffset=0cm]{geometry}
\pagestyle{empty}
\setlength{\parskip}{10pt}
%\setmainfont{Times New Roman}



\begin{document}
	
	\begin{center}
		МИНИСТЕРСТВО ОБРАЗОВАНИЯ И НАУКИ РОССИЙСКОЙ ФЕДЕРАЦИИ \\ ГОСУДАРСТВЕННОЕ БЮДЖЕТНОЕ ОБРАЗОВАТЕЛЬНОЕ УЧРЕЖДЕНИЕ \\ 
		ВЫСШЕГО ПРОФЕССИОНАЛЬНОГО ОБРАЗОВАНИЯ
		\vskip 1.5cm
		«Московский государственный технический \\
		университет имени Н.Э. Баумана» \\
		(МГТУ им. Н.Э. Баумана)
		\vskip 1.5cm
		ФАКУЛЬТЕТ ФУНДАМЕНТАЛЬНЫЕ НАУКИ \\
		КАФЕДРА \\
		«ВЫЧИСЛИТЕЛЬНАЯ МАТЕМАТИКА И МАТЕМАТИЧЕСКАЯ ФИЗИКА»
		\vskip 0.4cm
		Направление: \textbf{Математика и компьютерные науки}
		\vskip 0.4cm
		Дисциплина: Численные методы
		\vskip 0.4cm
		Домашняя работа №1-4 \\
		«Итерационный метод Якоби для полного решения задачи вычисления собственных значений
		и собственных векторов квадратной симметричной матрицы» \\
		Группа ФН11-51Б
		\vskip 0.2cm
		Вариант 1
		
		
		\vskip 1.5cm
		\begin{flushright}
			Студент: Авилов О.Д.
			
			\vskip 1.5cm
			
			Преподаватель: Кутыркин В.А
		\end{flushright}
		Оценка:
		
		\begin{figure}[b]
			\begin{center}
				Москва 2022
			\end{center}
		\end{figure}
		
	\end{center}
	
	% Конец титульника
	\newpage
\begin{center}
	\textbf{\textit{Задание}}
\end{center}
Используя метод Якоби, найти приближённое полное решение спектральной задачи для матрицы
$A$, приведённой в таблицах ниже. Останов выбрать на том шаге итерации, когда максимальная по
модулю внедиагональная компонента преобразованной матрицы станет меньше $\varepsilon=0.01$. 
Проверить найденные приближённые собственные векторы и отвечающие им собственные
значения матрицы $A$, проверив соответствующие приближённые равенства $(A \cdot {}^\mathsmaller{>} \tilde{q}_i \approx \tilde \lambda_i \cdot {}^\mathsmaller{>} \tilde{q}_i$ для любого $i \in \overline{1,4})$ с указанием погрешности.

$N = 1, \beta = 1-0,1(50-51) = 1,1$

\begin{equation*}
	A = A[0] = \left(
	\begin{array}{cccc}
		10\beta & 1 & 2 & 3\\
		1 & 10\beta & 3 & 2\\
		2 & 3 & 10\beta & 1\\
		3 & 2 & 1 & 10\beta\\
	\end{array}
	\right)
	=
	\left(
	\begin{array}{cccc}
		11 & 1 & 2 & 3\\
		1 & 11 & 3 & 2\\
		2 & 3 & 11 & 1\\
		3 & 2 & 1 & 11\\
	\end{array}
	\right)
\end{equation*}
\begin{center}
	\textbf{\textit{Решение}}
\end{center}
\begin{enumerate}[label=\textbf{\arabic*}] 
\item \textbf{Итерация} \\
	$a_1^4=3$ - максимальный по модулю внедиагональный элемент матрицы $A[0]$. \\
	Выберем угол поворота $\varphi = \dfrac\pi4$.
	Матрица поворота:
\begin{equation*}
	Q_0 = \left(
	\begin{array}{cccc}
		cos \varphi & 0 & 0 & -sin \varphi \\
		0 & 1 & 0 & 0 \\
		0 & 0 & 1 & 0 \\
		sin \varphi & 0 & 0 & cos \varphi \\
	\end{array}
	\right)
	=
	\left(
	\begin{array}{cccc}
		0.70711 & 0 & 0 & -0.7071 \\
		0 & 1 & 0 & 0 \\
		0 & 0 & 1 & 0 \\
		0.70711 & 0 & 0 & 0.70711 \\
	\end{array}
	\right)
\end{equation*}
\begin{equation*}
	{}^T Q_0 = \left(
	\begin{array}{cccc}
		0.70711 & 0 & 0 & 0.70711 \\
		0 & 1 & 0 & 0 \\
		0 & 0 & 1 & 0 \\
		-0.7071 & 0 & 0 & 0.70711 \\
	\end{array}
	\right)
\end{equation*}
\begin{equation*}
	A[1] = {}^T Q_0 \cdot A[0] \cdot Q_0 = \left(
	\begin{array}{cccc}
		14 & 2.12132 & 2.12132 & 0 \\
		2.12132 & 11 & 3 & 0.70711 \\
		2.12132 & 3 & 11 & -0.7071 \\
		0 & 0.70711 & -0.7071 & 8 \\
	\end{array}
	\right)
\end{equation*}
\item \textbf{Итерация} \\
$a_2^3=3$ - максимальный по модулю внедиагональный элемент матрицы $A[1]$. \\
Выберем угол поворота $\varphi = \dfrac\pi4$.
Матрица поворота:
\begin{equation*}
	Q_1 = \left(
	\begin{array}{cccc}
		1 & 0 & 0 & 0 \\
		0 & cos \varphi & -sin \varphi & 0 \\
		0 & sin \varphi & cos \varphi & 0 \\
		0 & 0 & 0 & 1 \\
	\end{array}
	\right)
	=
	\left(
	\begin{array}{cccc}
		1 & 0 & 0 & 0 \\
		0 & 0.70711 & -0.7071 & 0 \\
		0 & 0.70711 & 0.70711 & 0 \\
		0 & 0 & 0 & 1 \\
	\end{array}
	\right)
\end{equation*}
\begin{equation*}
	{}^T Q_1 = \left(
	\begin{array}{cccc}
		1 & 0 & 0 & 0 \\
		0 & 0.70711 & 0.70711 & 0 \\
		0 & -0.7071 & 0.70711 & 0 \\
		0 & 0 & 0 & 1 \\
	\end{array}
	\right)
\end{equation*}
\begin{equation*}
	A[2] = {}^T Q_1 \cdot A[1] \cdot Q_1 = \left(
	\begin{array}{cccc}
		14 & 3 & 0 & 0 \\
		3 & 14 & 0 & 0 \\
		0 & 0 & 8 & -1 \\
		0 & 0 & -1 & 8 \\
	\end{array}
	\right)
\end{equation*}
\item \textbf{Итерация} \\
$a_1^2=3$ - максимальный по модулю внедиагональный элемент матрицы $A[2]$. \\
Выберем угол поворота $\varphi = \dfrac\pi4$.
Матрица поворота:
\begin{equation*}
	Q_2 = \left(
	\begin{array}{cccc}
		cos \varphi & -sin \varphi & 0 & 0 \\
		sin \varphi & cos \varphi & 0 & 0 \\
		0 & 0 & 1 & 0 \\
		0 & 0 & 0 & 1 \\
	\end{array}
	\right)
	=
	\left(
	\begin{array}{cccc}
		0.70711 & -0.7071 & 0 & 0 \\
		0.70711 & 0.70711 & 0 & 0 \\
		0 & 0 & 1 & 0 \\
		0 & 0 & 0 & 1 \\
	\end{array}
	\right)
\end{equation*}
\begin{equation*}
	{}^T Q_2 = \left(
	\begin{array}{cccc}
		0.70711 & 0.70711 & 0 & 0 \\
		-0.7071 & 0.70711 & 0 & 0 \\
		0 & 0 & 1 & 0 \\
		0 & 0 & 0 & 1 \\
	\end{array}
	\right)
\end{equation*}
\begin{equation*}
	A[3] = {}^T Q_2 \cdot A[2] \cdot Q_2 = \left(
	\begin{array}{cccc}
		17 & 0 & 0 & 0 \\
		0 & 11 & 0 & 0 \\
		0 & 0 & 8 & -1 \\
		0 & 0 & -1 & 8 \\
	\end{array}
	\right)
\end{equation*}
\item \textbf{Итерация} \\
$a_3^4=3$ - максимальный по модулю внедиагональный элемент матрицы $A[3]$. \\
Выберем угол поворота $\varphi = \dfrac\pi4$.
Матрица поворота:
\begin{equation*}
	Q_3 = \left(
	\begin{array}{cccc}
		1 & 0 & 0 & 0 \\
		0 & 1 & 0 & 0 \\
		0 & 0 & cos \varphi & -sin \varphi \\
		0 & 0 & sin \varphi & cos \varphi \\
	\end{array}
	\right)
	=
	\left(
	\begin{array}{cccc}
		1 & 0 & 0 & 0 \\
		0 & 1 & 0 & 0 \\
		0 & 0 & 0.70711 & -0.7071 \\
		0 & 0 & 0.70711 & 0.70711 \\
	\end{array}
	\right)
\end{equation*}
\begin{equation*}
	{}^T Q_3 = \left(
	\begin{array}{cccc}
		1 & 0 & 0 & 0 \\
		0 & 1 & 0 & 0 \\
		0 & 0 & 0.70711 & 0.70711 \\
		0 & 0 & -0.7071 & 0.70711 \\
	\end{array}
	\right)
\end{equation*}
\begin{equation*}
	A[4] = {}^T Q_3 \cdot A[3] \cdot Q_3 = \left(
	\begin{array}{cccc}
		17 & 0 & 0 & 0 \\
		0 & 11 & 0 & 0 \\
		0 & 0 & 7 & 0 \\
		0 & 0 & 0 & 9 \\
	\end{array}
	\right)
\end{equation*}
\end{enumerate}
После 4 итерации максимальный по модулю среди всех внедиагональных элементов оказался меньше $\varepsilon = 0.01$ На диагонали матрицы $A[4]$ стоят все собственные значения матрицы $A$.
\begin{equation*}
	Q[3] = Q_0 \cdot Q_1 \cdot Q_2 \cdot Q_3= \left(
	\begin{array}{cccc}
		0.5 & -0.5 & -0.5 & -0.5 \\
		0.5 & 0.5 & -0.5 & 0.5 \\
		0.5 & 0.5 & 0.5 & -0.5 \\
		0.5 & -0.5 & 0.5 & 0.5 \\
	\end{array}
	\right)
\end{equation*}
Собственные векторы матрицы $A$ - это столбцы матрицы $Q[3]$
\begin{equation*}
	A \cdot {}^\mathsmaller{>} {q}_1 = \left(
	\begin{array}{c}
		8.5 \\ 8.5 \\ 8.5 \\ 8.5 \\ 
	\end{array}
	\right) = 
	\lambda_1 \cdot {}^\mathsmaller{>} {q}_1
\end{equation*}
\begin{equation*}
	A \cdot {}^\mathsmaller{>} {q}_2 = \left(
	\begin{array}{c}
		-5.5 \\ 5.5 \\ 5.5 \\ -5.5 \\  
	\end{array}
	\right) = 
	\lambda_2 \cdot {}^\mathsmaller{>} {q}_2
\end{equation*}
\begin{equation*}
	A \cdot {}^\mathsmaller{>} {q}_3 = \left(
	\begin{array}{c}
		-3.5 \\ -3.5 \\ 3.5 \\ 3.5 \\ 
	\end{array}
	\right) = 
	\lambda_3 \cdot {}^\mathsmaller{>} {q}_3
\end{equation*}
\begin{equation*}
	A \cdot {}^\mathsmaller{>} {q}_4 = \left(
	\begin{array}{c}
		-4.5 \\ 4.5 \\ -4.5 \\ 4.5 \\
	\end{array}
	\right) = 
	\lambda_4 \cdot {}^\mathsmaller{>} {q}_4
\end{equation*}
Погрешность моих вычислений вышла очень малой ($\varepsilon$ $\approx 0$).
\begin{center}
	\textbf{\textit{Результаты}}
\end{center}
Таким образом, используя метод Якоби, было найдено приближённое решение спектральной задачи матрицы $A$. Приближённые собственные значения данной матрицы стоят на главной диагонали матрицы $A[4]$. А её собственные значения являются столбцами матрицы $Q[3]$. При проверке соответствующих приближённых равенств погрешность вышла меньшей, чем смогла выдать написанная мной программа на python, то есть практически нулевой.
\end{document}