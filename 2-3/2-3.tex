\documentclass[12pt]{article}
%\usepackage{fontspec}
\usepackage{amsmath}
\usepackage{amsfonts}
\usepackage{graphicx}
\usepackage{enumitem}
\usepackage{amssymb}
\usepackage{relsize}
\usepackage{ upgreek }
\usepackage[utf8]{inputenc}
\usepackage[russian]{babel}
\usepackage[left=2cm,right=2cm,
top=2cm,bottom=2cm,bindingoffset=0cm]{geometry}
\pagestyle{empty}
\setlength{\parskip}{10pt}
%\setmainfont{Times New Roman}



\begin{document}
	
	\begin{center}
		МИНИСТЕРСТВО ОБРАЗОВАНИЯ И НАУКИ РОССИЙСКОЙ ФЕДЕРАЦИИ \\ ГОСУДАРСТВЕННОЕ БЮДЖЕТНОЕ ОБРАЗОВАТЕЛЬНОЕ УЧРЕЖДЕНИЕ \\ 
		ВЫСШЕГО ПРОФЕССИОНАЛЬНОГО ОБРАЗОВАНИЯ
		\vskip 1.5cm
		«Московский государственный технический \\
		университет имени Н.Э. Баумана» \\
		(МГТУ им. Н.Э. Баумана)
		\vskip 1.5cm
		ФАКУЛЬТЕТ ФУНДАМЕНТАЛЬНЫЕ НАУКИ \\
		КАФЕДРА \\
		«ВЫЧИСЛИТЕЛЬНАЯ МАТЕМАТИКА И МАТЕМАТИЧЕСКАЯ ФИЗИКА»
		\vskip 0.4cm
		Направление: \textbf{Математика и компьютерные науки}
		\vskip 0.4cm
		Дисциплина: Численные методы
		\vskip 0.4cm
		Домашняя работа №2-3 \\
		«Интегральное уравнение Фредгольма 2-го рода» \\
	Группа ФН11-51Б
	\vskip 0.2cm
	Вариант 1
	
	
	\vskip 1.5cm
	\begin{flushright}
		Студент: Авилов О.Д.
		
		\vskip 1.5cm
		
		Преподаватель: Кутыркин В.А
	\end{flushright}
	Оценка:
	
	\begin{figure}[b]
		\begin{center}
			Москва 2022
		\end{center}
	\end{figure}
	
\end{center}
	
	% Конец титульника
	\newpage
\begin{center}
	\textbf{\textit{Задание}}
\end{center}

\indentРассмотрим на квадрате 
$[a;b]\times[a;b]$
интегральное уравнение Фредгольма 2-го рода с 
симметричным, непрерывным и аналитически заданным ядром:
\begin{equation}
    x(s)-\lambda\int\limits_a^b K(s,\tau)x(\tau)d\tau=y(s),s\in[a;b]
\end{equation}
\indentИспользуя дискретный аналог уравнения (1), индуцированный методом конечных сумм с 
квадратурными формулами прямоугольников (количество узлов в квадратурной формуле 
не менее 20), найти приближённое решение уравнения (1), которое имеет вид:
$$x(s)-\frac{1}{1+\frac{n-45}{2}}\cdot\int\limits_0^{\frac{N+7}{N}} K(s,\tau)x(\tau)d\tau=\frac{N+3}{N}(s^2+\frac{n-53}{2}),s\in[0;\frac{N+7}{N}],$$
где N - номер студента в журнале, n - номер группы и
\begin{equation*}
K(s,\tau) = 
 \begin{cases}
   s\cdot(2\frac{N+7}{N}-\tau), &\text{$0\leq s\leq \tau$;}\\
   \tau\cdot(2\frac{N+7}{N}-s), &\text{$\tau \leq s\leq \frac{N+7}{N}$.}
 \end{cases}
\end{equation*}
\indentНайти ещё одно приближённое решение уравнения, имеющее вид частичной суммы ряда Фурье по ортонормированному базису из собственных функций интегрального оператора уравнения, используя три наименьшие по абсолютной величине характеристических числа интегрального оператора уравнения.\\
\indentВ узлах сетки сравнить значения приближённых решений по абсолютной величине,  проиллюстрировав это сравнение на соответствующем графике.
\begin{center}
	\textbf{\textit{Решение}}
\end{center}
N = 1, n = 51

\textbf{1. Метод конечных сумм для решения интегрального уравнения Фредгольма 2-го рода с симметричным, непрерывным и аналитически заданным ядром}\\
\indentДля построения дискретного аналога, аппроксимирующего уравнение (1), зададим на квадрате $[a;b]\times[a;b]$ двумерную центрально-равномерную сетку $B\times A=\langle(s_i,\tau_i):s_i\in B,\tau_i\in A\rangle$
типа $n\times n$ шага $(h,\tau)$. Следовательно, $B=\langle s_1,s_2,...,s_n\rangle$ и $A=\langle \tau_1,\tau_2,...,\tau_n\rangle$ центрально-равномерные сетки отрезка $[a;b]\times[a;b]$ с шагами $h=\frac{b-a}{n}$ и $\tau=\frac{b-a}{n}$, соответственно.\\

Составим центрально-равномерную с количеством узлов $m=20$.
\newpage
\indentЦентрально-равномерная сетка:

\begin{equation*}
    A=B=\left(
    \begin{array}{cccccccccccccccccccc}
0.2 \\ 0.6 \\ 1.0 \\ 1.4 \\ 1.8 \\ 2.2 \\ 2.6 \\ 3.0 \\ 3.4 \\ 3.8 \\ 4.2 \\ 4.6 \\ 5.0 \\ 5.4 \\ 5.8 \\ 6.2 \\ 6.6 \\ 7.0 \\ 7.4 \\ 7.8 \\

    \end{array}
    \right)
\end{equation*}

\indentСоставим матрицу $K(s,\tau)$:
\begin{equation*}
    K(s,\tau)=\left(
    \begin{array}{cccccccccccccccccccc}
3.16 & 3.08 & 3.0 & 2.92 & 2.84 & \ldots & 1.96 & 1.88 & 1.8 & 1.72 & 1.64 \\
3.08 & 9.24 & 9.0 & 8.76 & 8.52 & \ldots & 5.88 & 5.64 & 5.4 & 5.16 & 4.92 \\
3.0 & 9.0 & 15.0 & 14.6 & 14.2 & \ldots & 9.8 & 9.4 & 9.0 & 8.6 & 8.2 \\
2.92 & 8.76 & 14.6 & 20.44 & 19.88 & \ldots & 13.72 & 13.16 & 12.6 & 12.04 & 11.48 \\
2.84 & 8.52 & 14.2 & 19.88 & 25.56 & \ldots & 17.64 & 16.92 & 16.2 & 15.48 & 14.76 \\
2.76 & 8.28 & 13.8 & 19.32 & 24.84 & \ldots & 21.56 & 20.68 & 19.8 & 18.92 & 18.04 \\
2.68 & 8.04 & 13.4 & 18.76 & 24.12 & \ldots & 25.48 & 24.44 & 23.4 & 22.36 & 21.32 \\
2.6 & 7.8 & 13.0 & 18.2 & 23.4 & \ldots & 29.4 & 28.2 & 27.0 & 25.8 & 24.6 \\
2.52 & 7.56 & 12.6 & 17.64 & 22.68 & \ldots & 33.32 & 31.96 & 30.6 & 29.24 & 27.88 \\
2.44 & 7.32 & 12.2 & 17.08 & 21.96 & \ldots & 37.24 & 35.72 & 34.2 & 32.68 & 31.16 \\
2.36 & 7.08 & 11.8 & 16.52 & 21.24 & \ldots & 41.16 & 39.48 & 37.8 & 36.12 & 34.44 \\
2.28 & 6.84 & 11.4 & 15.96 & 20.52 & \ldots & 45.08 & 43.24 & 41.4 & 39.56 & 37.72 \\
2.2 & 6.6 & 11.0 & 15.4 & 19.8 & \ldots & 49.0 & 47.0 & 45.0 & 43.0 & 41.0 \\
2.12 & 6.36 & 10.6 & 14.84 & 19.08 & \ldots & 52.92 & 50.76 & 48.6 & 46.44 & 44.28 \\
2.04 & 6.12 & 10.2 & 14.28 & 18.36 & \ldots & 56.84 & 54.52 & 52.2 & 49.88 & 47.56 \\
1.96 & 5.88 & 9.8 & 13.72 & 17.64 & \ldots & 60.76 & 58.28 & 55.8 & 53.32 & 50.84 \\
1.88 & 5.64 & 9.4 & 13.16 & 16.92 & \ldots & 58.28 & 62.04 & 59.4 & 56.76 & 54.12 \\
1.8 & 5.4 & 9.0 & 12.6 & 16.2 & \ldots & 55.8 & 59.4 & 63.0 & 60.2 & 57.4 \\
1.72 & 5.16 & 8.6 & 12.04 & 15.48 & \ldots & 53.32 & 56.76 & 60.2 & 63.64 & 60.68 \\
1.64 & 4.92 & 8.2 & 11.48 & 14.76 & \ldots & 50.84 & 54.12 & 57.4 & 60.68 & 63.96 \\
    \end{array}
    \right)
\end{equation*}

\indent Решим СЛАУ: $$\mathbf{F\cdot{}^>x={}^>y},$$\indentгде $${}^>\mathbf x=[x^1,...,x^n\rangle, {}^>\mathbf y=[y^1,...,y^n\rangle\in{}^>\mathbb{R}^n, \mathbf F=(\delta^i_j-\lambda K^i_j\cdot h)^n_n=(f^i_j)^n_n\in L(\mathbb R,n)$$
\indent Матрица $F$:
\begin{equation*}
    F = \left(
    \begin{array}{cccccccccccccccccccc}
0.684 & -0.308 & -0.3 & -0.292 & -0.284 & \ldots & -0.196 & -0.188 & -0.18 & -0.172 & -0.164 \\
-0.308 & 0.076 & -0.9 & -0.876 & -0.852 & \ldots & -0.588 & -0.564 & -0.54 & -0.516 & -0.492 \\
-0.3 & -0.9 & -0.5 & -1.46 & -1.42 & \ldots & -0.98 & -0.94 & -0.9 & -0.86 & -0.82 \\
-0.292 & -0.876 & -1.46 & -1.044 & -1.988 & \ldots & -1.372 & -1.316 & -1.26 & -1.204 & -1.148 \\
-0.284 & -0.852 & -1.42 & -1.988 & -1.556 & \ldots & -1.764 & -1.692 & -1.62 & -1.548 & -1.476 \\
-0.276 & -0.828 & -1.38 & -1.932 & -2.484 & \ldots & -2.156 & -2.068 & -1.98 & -1.892 & -1.804 \\
-0.268 & -0.804 & -1.34 & -1.876 & -2.412 & \ldots & -2.548 & -2.444 & -2.34 & -2.236 & -2.132 \\
-0.26 & -0.78 & -1.3 & -1.82 & -2.34 & \ldots & -2.94 & -2.82 & -2.7 & -2.58 & -2.46 \\
-0.252 & -0.756 & -1.26 & -1.764 & -2.268 & \ldots & -3.332 & -3.196 & -3.06 & -2.924 & -2.788 \\
-0.244 & -0.732 & -1.22 & -1.708 & -2.196 & \ldots & -3.724 & -3.572 & -3.42 & -3.268 & -3.116 \\
-0.236 & -0.708 & -1.18 & -1.652 & -2.124 & \ldots & -4.116 & -3.948 & -3.78 & -3.612 & -3.444 \\
-0.228 & -0.684 & -1.14 & -1.596 & -2.052 & \ldots & -4.508 & -4.324 & -4.14 & -3.956 & -3.772 \\
-0.22 & -0.66 & -1.1 & -1.54 & -1.98 & \ldots & -4.9 & -4.7 & -4.5 & -4.3 & -4.1 \\
-0.212 & -0.636 & -1.06 & -1.484 & -1.908 & \ldots & -5.292 & -5.076 & -4.86 & -4.644 & -4.428 \\
-0.204 & -0.612 & -1.02 & -1.428 & -1.836 & \ldots & -5.684 & -5.452 & -5.22 & -4.988 & -4.756 \\
-0.196 & -0.588 & -0.98 & -1.372 & -1.764 & \ldots & -5.076 & -5.828 & -5.58 & -5.332 & -5.084 \\
-0.188 & -0.564 & -0.94 & -1.316 & -1.692 & \ldots & -5.828 & -5.204 & -5.94 & -5.676 & -5.412 \\
-0.18 & -0.54 & -0.9 & -1.26 & -1.62 & \ldots & -5.58 & -5.94 & -5.3 & -6.02 & -5.74 \\
-0.172 & -0.516 & -0.86 & -1.204 & -1.548 & \ldots & -5.332 & -5.676 & -6.02 & -5.364 & -6.068 \\
-0.164 & -0.492 & -0.82 & -1.148 & -1.476 & \ldots & -5.084 & -5.412 & -5.74 & -6.068 & -5.396 \\

    \end{array}
    \right)
\end{equation*}

\indent Обратная матрица:
\begin{equation*}
   F^{-1} =  \left(
    \begin{array}{cccccccccccccccccccc}
1.43656 & 0.39028 & 0.09423 & -0.2621 & \ldots & 0.43195 & 0.1894 & -0.1744 & -0.4265 \\
0.39028 & 1.92107 & 0.22237 & -0.6186 & \ldots & 1.01939 & 0.44699 & -0.4115 & -1.0066 \\
0.09423 & 0.22237 & 1.2082 & -0.5792 & \ldots & 0.95443 & 0.41851 & -0.3853 & -0.9425 \\
-0.2621 & -0.6186 & -0.5792 & 0.83091 & \ldots & 0.27863 & 0.12218 & -0.1125 & -0.2751 \\
-0.4507 & -1.0637 & -0.9959 & -0.2907 & \ldots & -0.5755 & -0.2523 & 0.2323 & 0.56828 \\
-0.3509 & -0.828 & -0.7753 & -0.2263 & \ldots & -1.0613 & -0.4654 & 0.4284 & 1.04799 \\
-0.0264 & -0.0624 & -0.0584 & -0.0171 & \ldots & -0.8679 & -0.3806 & 0.35032 & 0.85699 \\
0.31491 & 0.74318 & 0.69582 & 0.20313 & \ldots & -0.119 & -0.0522 & 0.04804 & 0.11752 \\
0.45471 & 1.07311 & 1.00472 & 0.29331 & \ldots & 0.70602 & 0.30958 & -0.285 & -0.6972 \\
0.3035 & 0.71625 & 0.67061 & 0.19577 & \ldots & 1.0792 & 0.47322 & -0.4356 & -1.0657 \\
-0.042 & -0.099 & -0.0927 & -0.0271 & \ldots & 0.76169 & 0.33399 & -0.3075 & -0.7521 \\
-0.3606 & -0.8509 & -0.7967 & -0.2326 & \ldots & -0.0433 & -0.019 & 0.01748 & 0.04276 \\
-0.4484 & -1.0582 & -0.9908 & -0.2892 & \ldots & -0.8206 & -0.3598 & 0.33123 & 0.81029 \\
-0.2493 & -0.5883 & -0.5508 & -0.1608 & \ldots & -1.0727 & -0.4704 & 0.43299 & 1.05923 \\
0.10939 & 0.25817 & 0.24171 & 0.07056 & \ldots & -0.6383 & -0.2799 & 0.25764 & 0.63027 \\
0.39804 & 0.93938 & 0.87952 & 0.25676 & \ldots & 0.20463 & 0.08973 & -0.0826 & -0.2021 \\
0.43195 & 1.01939 & 0.95443 & 0.27863 & \ldots & 1.91657 & 0.40191 & -0.37 & -0.9051 \\
0.1894 & 0.44699 & 0.41851 & 0.12218 & \ldots & 0.40191 & 1.45687 & -0.4206 & -1.0288 \\
-0.1744 & -0.4115 & -0.3853 & -0.1125 & \ldots & -0.37 & -0.4206 & 0.798 & -0.4941 \\
-0.4265 & -1.0066 & -0.9425 & -0.2751 & \ldots & -0.9051 & -1.0288 & -0.4941 & 1.3568 \\


    \end{array}
    \right)
\end{equation*}

\begin{equation*}
    {}^>y=\left(
    \begin{array}{cccccccccccccccccccc}
-3.84 \\ -2.56 \\ 0.0 \\ 3.84 \\ 8.96 \\ 15.36 \\ 23.04 \\ 32.0 \\ 42.24 \\ 53.76 \\ 66.56 \\ 80.64 \\ 96.0 \\ 112.64 \\ 130.56 \\ 149.76 \\ 170.24 \\ 192.0 \\ 215.04 \\ 239.36 \\

    \end{array}
    \right)
\end{equation*}

\indent Получили приближённое решение:
\begin{equation*}
    {}^>x=F^{-1} \cdot {}^>y = \left(
    \begin{array}{cccccccccccccccccccc}
-30.7449 \\ -63.5981 \\ -54.4684 \\ -9.199 \\ 43.2378 \\ 69.2824 \\ 52.26627 \\ 3.07972 \\ -46.7978 \\ -65.4448 \\ -40.9271 \\ 11.06397 \\ 57.25407 \\ 68.08157 \\ 36.61686 \\ -17.0026 \\ -58.4604 \\ -61.2236 \\ -23.5236 \\ 30.51147 \\


    \end{array}
    \right)
\end{equation*}

\textbf{2. Приближённое решение интегрального уравнения Фредгольма 2-го рода в виде 
частичной суммы ряда Фурье по ортонормированному базису из собственных 
функций интегрального оператора уравнения (используется три наименьшие по 
абсолютной величине характеристических числа интегрального оператора 
уравнения)}\\
\indentНайти специальное приближённое решение интегрального уравнения:
$$x(s)-\frac{1}{4}\cdot\int\limits_{0}^{8}K(s,\tau)x(\tau)d\tau=4(s^2-1), s\in[0;8]$$
\begin{equation*}
K(s,\tau) = 
 \begin{cases}
   s(16-\tau), &0\leq s\leq \tau;\\\
   \tau(16-s), &\tau\leq s\leq 16.
 \end{cases}
\end{equation*}
\indent Решим однородное уравнение
$$\varphi(s)-\lambda\int\limits_{0}^{8}K(s,\tau)\varphi(\tau)d\tau=0$$
\indent представим в виде:
$$\varphi(s)=\lambda(16-s)\int\limits_{0}^{s}\tau\varphi(\tau)d\tau +
\lambda s\int\limits_{s}^{8}(16-\tau)\varphi(\tau)d\tau$$
\indent Продифференцируем это равенство:
$$\varphi^{'}(s)=-\lambda\int\limits_{0}^{s}\tau\varphi(\tau)d\tau
+\lambda\int\limits_{s}^{8}(16-\tau)\varphi(\tau)d\tau$$
\indent Снова продифференцируем:
$$\varphi^{''}(s)=-\lambda s \varphi(s) - \lambda (16-s) \varphi(s) = -16\lambda\varphi(s) $$
Так как $\lambda = \dfrac 1 4$:
$$\varphi^{''}(s)+4\varphi(s)=0$$
\indent Определим граничные условия:
$$\varphi(0)=0$$
$$\left.
  \begin{array}{ccc}
    \varphi(8) & = & 	
    8 \lambda \int\limits_{0}^{8}\tau\varphi(\tau)d\tau \\
    \varphi^{'}(8) & = & -\lambda\int\limits_{0}^{8}\tau\varphi(\tau)d\tau \\
  \end{array}
\right\} \Rightarrow \varphi(8)+8\varphi^{'}(8)=0$$
\indentПолучаем краевую задачу:
\begin{equation*}
	\begin{cases}
		\varphi^{''}(s)+4\varphi(s)=0
		\\
		\varphi(0)=0 \\
		\varphi(8)+8\varphi^{'}(8)=0
	\end{cases}
\end{equation*}

3)Так как $\lambda>0$. Общее решение имеет вид $\varphi(s)=C_1\cdot \cos(4\sqrt{\lambda}s)+
C_2\cdot \sin(4 \sqrt{\lambda}s)$. Подставляем граничные условия:
\begin{equation*}
 \begin{cases}
   \varphi(0)=C_1=0; 
   \\
   \varphi(8)+8\varphi^{'}(8)=
   C_2(\sin(32\sqrt{\lambda})+
   32\sqrt{\lambda} \cos(32\sqrt{\lambda})=0
 \end{cases}
\end{equation*}
\indentДля нетривиального решения:
$$\sin(32\sqrt{\lambda})+
32\sqrt{\lambda} \cos(32\sqrt{\lambda}) = 0$$
$$32\sqrt{\lambda}=-tg(32\sqrt{\lambda})$$
\indentНайдём первые три корня этого равенства(найдено с помощью WolframAlpha):\\
$$\lambda_1=0,00402$$
$$\lambda_2=0,02357$$
$$\lambda_3=0,06217$$
\indentПри $\lambda = \lambda_i, i = \overline{0,2}$ система краевых условий принимает вид:
\begin{equation*}
	\begin{cases}
		C_1=0; 
		\\
		0 \cdot C_2 =0
	\end{cases}
\end{equation*}
\indentОна имеет бесконечное множество ненулевых решений:
\begin{equation*}
	\begin{cases}
		C_1=0; 
		\\
		C_2 = C
	\end{cases}
\end{equation*}
\indentВозьмем $C = 1$, тогда cобственные функции имеют вид:
$$\varphi_i(s)=\sin(4 \lambda_i \cdot s)$$
\indentПриступим к поиску $\alpha_n$ в разложении $f$ в ряд Фурье по ортонормированной 
системе $\varphi_i(s)$:\\
$$f=\alpha_0\varphi_0+\alpha_1\varphi_1+\alpha_2\varphi_2+u$$
$$f=\alpha_0 \sin(0.25359)+\alpha_1(0.61415s)+\alpha_2(0.99733s)+u$$
$$\alpha_0=\frac{\int\limits_{0}^{8}f\cdot\varphi_0ds}{\int\limits_{0}^{8}\varphi_0\cdot\varphi_0ds} = 127.3109$$
$$\alpha_1=\frac{\int\limits_{0}^{8}f\cdot\varphi_1ds}{\int\limits_{0}^{8}\varphi_1\cdot\varphi_1ds} = -67.868$$
$$\alpha_2=\frac{\int\limits_{0}^{8}f\cdot\varphi_2ds}{\int\limits_{0}^{8}\varphi_2\cdot\varphi_2ds} = 20.2341$$
\indentТаким образом, приближённое решение:
$$x=4(s^2-1)+\frac{1}{4}(-517.5646 \sin(0.25359s) + 299.7354 \sin(0.61415s)-107.7239 \sin(0.99733s))$$

% \begin{center}
%   \begin{tabular}{|c|c|с|c|}
%  \hline
%  Узел & Аналитическое & Приближённое & Погрешность\\
%  \hline
% 0,0409 & -1,2780 & -1,2727 & 0,0053 \\
% 0,1227 & -1,2688 & -1,2590 & 0,0098 \\
% 0,2045 & -1,2356 & -1,2257 & 0,0099 \\
% 0,2864 & -1,1786 & -1,1713 & 0,0073 \\
% 0,3682 & -1,0981 & -1,0946 & 0,0035 \\
% 0,4500 & -0,9945 & -0,9949 & 0,0004 \\
% 0,5318 & -0,8685 & -0,8719 & 0,0033 \\
% 0,6136 & -0,7207 & -0,7256 & 0,0049 \\
% 0,6955 & -0,5518 & -0,5568 & 0,0049 \\
% 0,7773 & -0,3630 & -0,3665 & 0,0036 \\
% 0,8591 & -0,1550 & -0,1562 & 0,0012 \\
% 0,9409 & 0,0708 & 0,0724 & 0,0016 \\
% 1,0227 & 0,3132 & 0,3174 & 0,0042 \\
% 1,1045 & 0,5710 & 0,5769 & 0,0059 \\
% 1,1864 & 0,8427 & 0,8490 & 0,0064 \\
% 1,2682 & 1,1268 & 1,1320 & 0,0052 \\
% 1,3500 & 1,4218 & 1,4244 & 0,0026 \\
% 1,4318 & 1,7260 & 1,7249 & 0,0011 \\
% 1,5136 & 2,0379 & 2,0329 & 0,0049 \\
% 1,5955 & 2,3556 & 2,3480 & 0,0076 \\

%  \hline
% \end{tabular}
% \end{center}

\begin{center}
  \begin{tabular}{|c|c|c|c|}
 \hline
 Узел & Метод конечных сумм(1 часть) & Приближённое(2 часть) & Погрешность\\
 \hline
0.2 & -30.7449 & -6.5551 & 24.18985 \\
0.6 & -63.5981 & -10.3511 & 53.24702 \\
1.0 & -54.4684 & -11.9036 & 42.56486 \\
1.4 & -9.199 & -10.8823 & 1.68334 \\
1.8 & 43.2378 & -7.3628 & 50.60056 \\
2.2 & 69.2824 & -1.8745 & 71.15685 \\
2.6 & 52.26627 & 4.64085 & 47.62542 \\
3.0 & 3.07972 & 10.95851 & 7.87879 \\
3.4 & -46.7978 & 15.76926 & 62.56711 \\
3.8 & -65.4448 & 17.91567 & 83.36046 \\
4.2 & -40.9271 & 16.62864 & 57.55571 \\
4.6 & 11.06397 & 11.72420 & 0.66022 \\
5.0 & 57.25407 & 3.72566 & 53.52842 \\
5.4 & 68.08157 & -6.1131 & 74.19464 \\
5.8 & 36.61686 & -15.8926 & 52.50948 \\
6.2 & -17.0026 & -23.2547 & 6.25203 \\
6.6 & -58.4604 & -25.6471 & 32.81331 \\
7.0 & -61.2236 & -20.6325 & 40.59111 \\
7.4 & -23.5236 & -6.1960 & 17.32758 \\
7.8 & 30.51147 & 18.98897 & 11.52249 \\

 \hline
\end{tabular}
\end{center}

\indent Максимальное значение абсолютной погрешности составляет 83.3605

\newpage
\indentПостроим совмещенный график решений:
\begin{figure}[h!]
	\centering
	\scalebox{1}{\includegraphics{graph.png}}
\end{figure}

\textbf{Вывод: }В ходе лабораторной работы было решено уравнение Фредгольма 2-го рода методом конечных сумм и в виде
частичной суммы ряда Фурье по ортонормированному базису из собственных
функций интегрального оператора уравнения. Графики приближённых решений не совпали ввиду большого шага сетки, а также (по моему мнению) получившихся очень маленьких значений характеристических чисел.







\end{document}