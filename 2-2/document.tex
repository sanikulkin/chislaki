\documentclass[12pt]{article}

\usepackage{amsmath}
\usepackage{amsfonts}
\usepackage{amssymb}
\usepackage{relsize}
\usepackage{ upgreek }
\usepackage[utf8]{inputenc}
\usepackage[russian]{babel}
\usepackage[left=2cm,right=2cm,
top=2cm,bottom=2cm,bindingoffset=0cm]{geometry}
\pagestyle{empty}
\setlength{\parskip}{10pt}
\usepackage{graphicx}
\graphicspath{ {./} }



\begin{document}
	
	\begin{center}
		МИНИСТЕРСТВО ОБРАЗОВАНИЯ И НАУКИ РОССИЙСКОЙ ФЕДЕРАЦИИ \\ ГОСУДАРСТВЕННОЕ БЮДЖЕТНОЕ ОБРАЗОВАТЕЛЬНОЕ УЧРЕЖДЕНИЕ \\ 
		ВЫСШЕГО ПРОФЕССИОНАЛЬНОГО ОБРАЗОВАНИЯ
		\vskip 1.5cm
		«Московский государственный технический \\
		университет имени Н.Э. Баумана» \\
		(МГТУ им. Н.Э. Баумана)
		\vskip 1.5cm
		ФАКУЛЬТЕТ ФУНДАМЕНТАЛЬНЫЕ НАУКИ \\
		КАФЕДРА \\
		«ВЫЧИСЛИТЕЛЬНАЯ МАТЕМАТИКА И МАТЕМАТИЧЕСКАЯ ФИЗИКА»
		\vskip 0.4cm
		Направление: \textbf{Математика и компьютерные науки}
		\vskip 0.4cm
		Дисциплина: Численные методы
		\vskip 0.4cm
		Домашняя работа №2-2 \\
		«Интерполяционные кубические сплайны» \\
		Группа ФН11-51Б
		\vskip 0.2cm
		Вариант 1
		
		
		\vskip 1.5cm
		\begin{flushright}
			Студент: Авилов О.Д.
			
			\vskip 1.5cm
			
			Преподаватель: Кутыркин В.А
		\end{flushright}
		Оценка:
		
		\begin{figure}[b]
			\begin{center}
				Москва 2022
			\end{center}
		\end{figure}
		
	\end{center}
	
	% Конец титульника
	\newpage
	\begin{center}
		\textbf{\textit{Задание}}
	\end{center}
На отрезке $[0,2]$ задана равномерная сетка $A = \langle \tau_0, \tau_1, \dots, \tau_{k} \rangle $, где $k = 10$, с шагом $h = 0.2 = stp(A)$ и определена функция
$f(\tau) = 2sin \left(\dfrac{ \pi \tau} 2 \right) \sqrt{2(58-n) + N \tau^2 \sqrt{23-N}}$, где $n$ - номер группы и $N$ - номер студента в журнале группы. Для $A$-сеточной функции ${}^{\mathsmaller{>}} y = \hat A(f) =  [y_0, y_1, \dots, y_{k} \rangle \in {}^{\mathsmaller{>}}{\mathbb R}^{|A|}(A)$, где $y_i = f(\tau_i)$ для $i = \overline{0,k}$, решить задачу $A$-интерполяции сеточной функции ${}^{\mathsmaller{>}} y$ с помощью сплайна $spl_3(A; {}^{\mathsmaller{>}} y)$ 3-ей степени дефекта 1. Затем сравнить в узлах равномерной сетки
$B = \left\langle \theta_0, \theta_1, \ldots, \theta_{2k} \right\rangle (stp(B) = 0.1)$ отрезка $[0,2]$ значения функции $f(\tau)$ и сплайна $spl_3(A; {}^{\mathsmaller{>}} y)$ и, кроме того, значения их
производных, т.е. значения функций $\dfrac{df}{d \tau}$ и $\dfrac{d spl_3(A; {}^{\mathsmaller{>}} y)}{d \tau}$. Результаты проиллюстрировать
графически.
 	\begin{center}
	\textbf{\textit{Решение}}
\end{center}
N = 1, n = 51
$$ f(\tau) = 2sin \left(\dfrac{ \pi \tau} 2 \right) \sqrt{14 + \tau^2 \sqrt{22}}
$$

На отрезке  $[0,2]$ равномерная сетка $A = \langle \tau_0, \tau_1, \dots, \tau_{10} \rangle $ и соответствующая ей сеточная функция ${}^{\mathsmaller{>}} y = [f(\tau_0), f(\tau_1), \dots, f(\tau_{10}) \rangle \in \underline{\mathbb R}^{11}(A) $ имеют вид:

$$ 
A =
\left(
\begin{array}{c}
	0.0 \\ 0.2 \\ 0.4 \\ 0.6 \\ 0.8 \\ 1.0 \\ 1.2 \\ 1.4 \\ 1.6 \\ 1.8 \\ 2.0
\end{array}
\right) ,
\hspace{0.2cm}
{}^{\mathsmaller{>}} y = 
\left(
\begin{array}{c}
0.0 \\ 2.32791 \\ 4.51494 \\ 6.40883 \\ 7.84304 \\ 8.64648 \\ 8.66541 \\ 7.79234 \\ 5.99512 \\ 3.3395 \\ 0.0
\end{array}
\right)
$$ 

Определим коэффициенты $a_i = y_{i-1}, 	i \in \langle 1, \ldots, k \rangle$ и $g_i = \dfrac{a_{i+1} - a_i}{h_i} , 	i \in \langle 1, \ldots, k - 1 \rangle$, где $h_i = \tau_i - \tau_{i-1}, i \in \langle 1, \ldots, k \rangle$

$$ 
a =
\left(
\begin{array}{c}
0.0 \\ 2.32791 \\ 4.51494 \\ 6.40883 \\ 7.84304 \\ 8.64648 \\ 8.66541 \\ 7.79234 \\ 5.99512 \\ 3.3395
\end{array}
\right) ,
\hspace{0.2cm}
g = 
\left(
\begin{array}{c}
	11.63957 \\ 10.9351 \\ 9.46949 \\ 7.17105 \\ 4.0172 \\ 0.09466 \\ -4.3653 \\ -8.9861 \\ -13.2781
\end{array}
\right)
$$ 

Для нахождения $c$ воспользуемся системой уравнений:
$$
\begin{cases}
	c_1 = 0 \\
	h_i c_i + 2(h_i + h_{i+1})c_{i+1} + h_{i+1} c_{i+2} = 3(g_{i+1} - g_i),  i \in \langle 1, \ldots, k - 2 \rangle \\
	c_k = 0
\end{cases}
$$

Решим второе уравнение, как СЛАУ вида $Z \hat c = l$, где:
$$
Z = \left(
\begin{array}{ccccccccccc}
	0.8 & 0.2 & 0.0 & 0.0 & 0.0 & 0.0 & 0.0 & 0.0 \\
	0.2 & 0.8 & 0.2 & 0.0 & 0.0 & 0.0 & 0.0 & 0.0 \\
	0.0 & 0.2 & 0.8 & 0.2 & 0.0 & 0.0 & 0.0 & 0.0 \\
	0.0 & 0.0 & 0.2 & 0.8 & 0.2 & 0.0 & 0.0 & 0.0 \\
	0.0 & 0.0 & 0.0 & 0.2 & 0.8 & 0.2 & 0.0 & 0.0 \\
	0.0 & 0.0 & 0.0 & 0.0 & 0.2 & 0.8 & 0.2 & 0.0 \\
	0.0 & 0.0 & 0.0 & 0.0 & 0.0 & 0.2 & 0.8 & 0.2 \\
	0.0 & 0.0 & 0.0 & 0.0 & 0.0 & 0.0 & 0.2 & 0.8 \\
\end{array}
\right), \hspace{0.2 cm}
l = 
\left(
\begin{array}{c}
	-2.1134 \\ -4.3968 \\ -6.8953 \\ -9.4615 \\ -11.7676 \\ -13.38 \\ -13.8624 \\ -12.8759 
\end{array}
\right)
$$
$$\hat c = Z^{-1}l = 
\left(
\begin{array}{c}
	-1.7342 \\ -3.6302 \\ -5.729 \\ -7.9303 \\ -9.8574 \\ -11.4781 \\ -11.1304 \\ -13.3123 
\end{array}
\right)
$$
Добавим $c_1$ и $c_k$:
$$
c = 
\left(
\begin{array}{c}
	0.0 \\ -1.7342 \\ -3.6302 \\ -5.729 \\ -7.9303 \\ -9.8574 \\ -11.4781 \\ -11.1304 \\ -13.3123 \\ 0.0
\end{array}
\right)
$$

Определим коэффициенты $b_i$ и $d_i$ по формулам:
$$
\begin{cases}
	b_i = g_i = \dfrac{h_i}{3}(2c_i + c_{i+1}),  i \in \langle 1, \ldots, k - 1 \rangle \\
	b_k = b_{k-1} + h_{k-1}(c_{k-1} + c_k)
\end{cases}
$$
$$
d_i = \dfrac{1}{3h_i}(c_{i+1} - c_i), i \in \langle 1, \ldots, k - 1 \rangle 
$$

$$ 
b =
\left(
\begin{array}{c}
	11.75519 \\ 11.40835 \\ 10.33546 \\ 8.46361 \\ 5.73173 \\ 2.17418 \\ -2.0929 \\ -6.6146 \\ -11.5031 \\ -12.3906
\end{array}
\right) ,
\hspace{0.2cm}
d = 
\left(
\begin{array}{c}
	-2.8903 \\ -3.16 \\ -3.498 \\ -3.6688 \\ -3.2119 \\ -2.7011 \\ 0.57953 \\ -3.6366 \\ 22.18715 \\ -107.6718
\end{array}
\right)
$$ 

Теперь мы можем найти  неизвестные коэффициенты полиномов $P_1, P_2, \ldots, P_k$ сплайна $spl_3(A; {}^{\mathsmaller{>}} y)$ по формуле:
$$
P_i(t) = a_i + b_i(t- \tau_{i-1}) + c_i(t- \tau_{i-1})^2 + d_i(t- \tau_{i-1})^3
$$
Построим графики $f(\tau)$ и $spl_3(A; {}^{\mathsmaller{>}} y)$ :
\begin{center}
	\includegraphics[scale=0.9]{1.png}
\end{center}

Производная функции $f(\tau)$:
$$f ^ \prime (\tau) = \pi \sqrt{\sqrt{22}\tau^2 + 14}cos(\dfrac{\pi \tau}{2}) + 
\dfrac {2 \sqrt{22} \tau sin(\dfrac{\pi \tau}{2})} {\sqrt{\sqrt{22}\tau^2 + 14}}
$$

Коэффициенты $P_1^\prime, P_2^\prime, \ldots, P_k^\prime$ производной сплайна $\dfrac {d spl_3(A; {}^{\mathsmaller{>}} y)} {d \tau}$:
$$
P_i^\prime(\tau_i) = b_i + 2 c_i(\tau_i - \tau_{i-1}) + 3d_i(\tau_i - \tau_{i-1})^2
$$

Построим графики $\dfrac {d f(\tau)} {d \tau}$ и $\dfrac {d spl_3(A; {}^{\mathsmaller{>}} y)} {d \tau}$ :
\begin{center}
	\includegraphics[scale=0.9]{2.png}
\end{center}
\begin{center}
	\textbf{\textit{Результаты}}
\end{center}

В результате лабораторной работы была решена задача интерполяции сеточной функции с помощью сплайна 3-ей степени единичного дефекта, а также было проведено сравнение значений функции и сплайна и их производных. Получили небольшое расхождение графиков производных, что вызвано большим шагом нашей сетки $A$.
\end{document}