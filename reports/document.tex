\documentclass[12pt]{article}
%\usepackage{fontspec}
\usepackage{amsmath}
\usepackage{amsfonts}
\usepackage{amssymb}
\usepackage{relsize}
\usepackage{ upgreek }
\usepackage[utf8]{inputenc}
\usepackage[russian]{babel}
\usepackage[left=2cm,right=2cm,
top=2cm,bottom=2cm,bindingoffset=0cm]{geometry}
\pagestyle{empty}
\setlength{\parskip}{10pt}
%\setmainfont{Times New Roman}



\begin{document}
	
	\begin{center}
		МИНИСТЕРСТВО ОБРАЗОВАНИЯ И НАУКИ РОССИЙСКОЙ ФЕДЕРАЦИИ \\ ГОСУДАРСТВЕННОЕ БЮДЖЕТНОЕ ОБРАЗОВАТЕЛЬНОЕ УЧРЕЖДЕНИЕ \\ 
		ВЫСШЕГО ПРОФЕССИОНАЛЬНОГО ОБРАЗОВАНИЯ
		\vskip 1.5cm
		«Московский государственный технический \\
		университет имени Н.Э. Баумана» \\
		(МГТУ им. Н.Э. Баумана)
		\vskip 1.5cm
		ФАКУЛЬТЕТ ФУНДАМЕНТАЛЬНЫЕ НАУКИ \\
		КАФЕДРА \\
		«ВЫЧИСЛИТЕЛЬНАЯ МАТЕМАТИКА И МАТЕМАТИЧЕСКАЯ ФИЗИКА»
		\vskip 0.4cm
		Направление: \textbf{Математика и компьютерные науки}
		\vskip 0.4cm
		Дисциплина: Численные методы
		\vskip 0.4cm
		Домашняя работа №1-1 \\
		«Погрешности при решении СЛАУ» \\
		Группа ФН11-51Б
		\vskip 0.2cm
		Вариант 1
		
		
		\vskip 1.5cm
		\begin{flushright}
			Студент: Авилов О.Д.
			
			\vskip 1.5cm
			
			Преподаватель: Кутыркин В.А
		\end{flushright}
		Оценка:
		
		\begin{figure}[b]
			\begin{center}
				Москва 2022
			\end{center}
		\end{figure}
		
	\end{center}

% Конец титульника
\newpage



	\begin{center}
		\textbf{\textit{Задание 1.1}}
	\end{center}
	Дана СЛАУ (N – номер студента в журнале, N = 10, $\alpha$ = (n - 50) / 100, где n - номер группы):
	\begin{equation}
		\begin{cases}
			50(1 + 0,5N + \alpha)x^1 + 50(1 + 0,5N)x^2 + 50(1 + 0,5N)x^3 = 50(3 + 1,5N + \alpha)\\
			50,1(1 + 0,5N)x^1 + 49,9(1 + 0,5N + \alpha)x^2 + 50(1 + 0,5N)x^3 = 50(3 + 1,5N + \alpha)\\
			49,9(1 + 0,5N)x^1 + 50(1 + 0,5N)x^2 + 50,1(1 + 0,5N + \alpha)x^3 = 50(3 + 1,5N + \alpha)\\
		\end{cases}
	\end{equation}
	Предполагается, что ошибка в матрице этой СЛАУ достаточно мала и относительная
	ошибка в её правой части равна 0,01. Приближённая СЛАУ имеет вид:
	\begin{equation}
		\begin{cases}
		50(1 + 0,5N + \alpha)x^1 + 50(1 + 0,5N)x^2 + 50(1 + 0,5N)x^3 = 50(3 + 1,5N + \alpha)(1 + 0,01)\\
			50,1(1 + 0,5N)x^1 + 49,9(1 + 0,5N + \alpha)x^2 + 50(1 + 0,5N)x^3 = 50(3 + 1,5N + \alpha)(1 - 0,01)\\
			49,9(1 + 0,5N)x^1 + 50(1 + 0,5N)x^2 + 50,1(1 + 0,5N + \alpha)x^3 = 50(3 + 1,5N + \alpha)(1 + 0,01)\\
		\end{cases}
	\end{equation}
	Требуется найти число обусловленности матрицы рассматриваемой СЛАУ и
	относительную погрешность в решении приближённой СЛАУ. Затем, прокомментировать
	получившиеся результаты.
	\begin{center}
		\textbf{\textit{Решение}}
	\end{center}
	Матрица СЛАУ (1) имеет вид
	\begin{equation*}
		A = \left(
		\begin{array}{ccc}
			75,5 & 75 & 75\\
			75,15 & 75,349 & 75 \\
			74,85 & 75 & 75,651 \\
		\end{array}
		\right)
	\end{equation*}

	1. Найдем число обусловленности матрицы A \\
	Число обусловленности матрицы равно
	\begin{equation*}
		cond\left(A\right)=||A|| \cdot ||A^{-1}||
	\end{equation*}
	Вычислим обратную матрицу с помощью функции МОБР
	\begin{equation*}
		A^{-1} = \left(
		\begin{array}{ccc}
			1,467675 & -0,95257 & -0,51067 \\
			-1,39345 & 1,910029 & -0,51213 \\
			-0,07068 & -0,95111	& 1,026211 \\
		\end{array}
		\right)
	\end{equation*}
Норма матрицы вычисляется по формуле
\begin{equation*}
	||A|| = \max\left\{ \sum_{j=1}^{n}|a^i_j|:i=\overline{1,n}\right\}
\end{equation*}
Для этого сначала вычислим матрицы, составленные из модулей элементов исходных
матриц, воспользовавшись функцией ABS
\begin{equation*}
	\left(
	\begin{array}{ccc}
		75,5 & 75 & 75\\
		75,15 & 75,349 & 75 \\
		74,85 & 75 & 75,651 \\
	\end{array}
	\right)
	,
	\hspace{0.5cm}
	\left(
	\begin{array}{ccc}
		1,467675 & 0,952571 & 0,510672 \\
		1,393449 & 1,910029 & 0,512135 \\
		0,070678 & 0,951108	& 1,026211 \\
	\end{array}
	\right)
\end{equation*}
Поскольку $||A||=225,501$ и $||A^{-1}||=3,815613$, число обусловленности $cond\left(A\right)=||A|| \cdot ||A^{-1}|| = 860,4245 $ \\
Таким образом, СЛАУ (1) плохо обусловлена

2. Найдем относительную погрешность в решении приближенной СЛАУ (2)
Запишем СЛАУ (1) и (2) в матричной форме

\begin{equation*}
	\left(
	\begin{array}{ccc}
		75,5 & 75 & 75\\
		75,15 & 75,349 & 75 \\
		74,85 & 75 & 75,651 \\
	\end{array}
	\right)
	\left(
	\begin{array}{c}
		x^1\\
		x^2 \\
		x^3 \\
	\end{array}
	\right)
	=
	\left(
	\begin{array}{c}
		225,951\\
		225,951\\
		225,951\\
	\end{array}
	\right)
\end{equation*}
\begin{equation*}
	\mathbf{^{\mathsmaller{>}} x=A^{-1} {}^\mathsmaller{>}b} = 
	\left(
	\begin{array}{ccc}
		1,467675 & -0,95257 & -0,51067 \\
		-1,39345 & 1,910029 & -0,51213 \\
		-0,07068 & -0,95111	& 1,026211 \\
	\end{array}
	\right)
	\left(
	\begin{array}{c}
		225,951\\
		225,951\\
		225,951\\
	\end{array}
	\right)
	=
	\left(
	\begin{array}{c}
		1,001557\\
		1,004427\\
		1,000019\\
	\end{array}
	\right)
\end{equation*}
\begin{equation*}
\left(
\begin{array}{ccc}
	75,5 & 75 & 75\\
	75,15 & 75,349 & 75 \\
	74,85 & 75 & 75,651 \\
\end{array}
\right)
	\left(
	\begin{array}{c}
		x^1\\
		x^2 \\
		x^3 \\
	\end{array}
	\right)
	=
	\left(
	\begin{array}{c}
		228,2105\\
		223,6915\\
		228,2105\\
	\end{array}
	\right)
\end{equation*}
Тогда ошибка:
\begin{equation*}
\mathbf{^{\mathsmaller{>}} \Delta x=A^{-1} {}^\mathsmaller{>} \Delta b} = 
	\left(
\begin{array}{ccc}
	1,467675 & -0,95257 & -0,51067 \\
	-1,39345 & 1,910029 & -0,51213 \\
	-0,07068 & -0,95111	& 1,026211 \\
\end{array}
\right)
\left(
\begin{array}{c}
	2,25951\\
	-2,25951\\
	2,25951\\
\end{array}
\right)
=
\left(
\begin{array}{c}
	4,314703\\
	-8,62142\\
	4,308075\\
\end{array}
\right)
\end{equation*}
Поскольку $||\mathbf{^{\mathsmaller{>}} x}|| = 1,004427$ и 
$||\mathbf{^{\mathsmaller{>}} \Delta x}|| = 8, 62142$ , относительная погрешность $\dfrac{||\mathbf{^{\mathsmaller{>}} \Delta x}||}{||\mathbf{^{\mathsmaller{>}} x}||} = 8,58342$. \\
Действительно, $\dfrac{||\mathbf{^{\mathsmaller{>}} \Delta x}||}{||\mathbf{^{\mathsmaller{>}} x}||} = 8,58342 \leqslant cond(A) 
\dfrac{||\mathbf{^{\mathsmaller{>}} \Delta b}||}{||\mathbf{^{\mathsmaller{>}} b}||} = 
860,4245 \cdot 0,01 = 8,604245$
\begin{center}
	\textbf{\textit{Результаты}}
\end{center}
Число обусловленности $cond\left(A\right)=||A|| \cdot ||A^{-1}|| = 860,4245 > 10^2 $ , значит, матрица СЛАУ плохо
обусловлена. Относительная погрешность решения $\dfrac{||\mathbf{^{\mathsmaller{>}} \Delta x}||}{||\mathbf{^{\mathsmaller{>}} x}||} = 8,58342 $ большая
вследствие плохой обусловленности СЛАУ.

\begin{center}
	\textbf{\textit{Задание 1.2}}
\end{center}
Исходные данные варианта 1: $\uplambda=0,51 \hspace{0.1cm} , \hspace{0.5cm}  F(x)=\sin{x},
\hspace{0.5cm} a=\dfrac{\pi}{4},
\hspace{0.5cm} b=\dfrac{3\pi}{4}$ \\ 

На отрезке [a;b] выбрана центрально равномерная сетка с десятью узлами 
$s_1 = \tau_1 = a + \dfrac{h}{2}, s_2 = \tau_2 =  \tau_1 + h
\dots, s_{10} = \tau_{10} = \tau_9 + h$,
имеющая шаг  $h= \dfrac{b-a}{10}$.

Требуется решить приближенную СЛАУ:

\begin{equation}
\mathbf{\left(E + \uplambda A  \right) {}^{\mathsmaller{>}} x = 
{}^{\mathsmaller{>}} b + {}^{\mathsmaller{>}} \Delta b ,}
\end{equation}
где $ \uplambda \in \mathbb{R} $ – ненулевое число, 
$ E \in GL(\mathbb{R};10) $ – единичная матрица,
$ A = (a^i_j)^{10}_{10} \in GL(\mathbb{R};10) $  и 
$ \mathbf{^{\mathsmaller{>}} b} = [b^1,\dots,b^{10}\rangle \in \mathbf{^{\mathsmaller{>}}}
\mathbb{R}^{10}$ - матрица и вектор, соответственно,
которые с помощью исходных данных определяются соотношениями: 

$ a^i_j=F(s_i \cdot \tau_j)\dfrac{b-a}{10}) $ для 
$ i,j = \overline{1,10}, 
\mathbf{
	{}^{\mathsmaller{>}} b = \left(E + \uplambda A  \right) {}^{\mathsmaller{>}} x_*}
$ и 
$ \mathbf{{}^{\mathsmaller{>}} x_*} = [1,1,\dots,1\rangle \in {}^{\mathsmaller{>}}
	\mathbb{R}^{10}
$

Согласно СЛАУ (3), приближённая СЛАУ определяется только погрешностью
$\mathbf{{}^{\mathsmaller{>}} \Delta b} = [\Delta b^1,\dots,\Delta b^{10}\rangle = 
0,01 \cdot [b^1,-b^2,b^3,-b^4,b^5,-b^6,b^7,-b^8,b^9,-b^{10}\rangle \in  
{}^{\mathsmaller{>}} \mathbb{R}^{10}$ в правой части
СЛАУ (3).

Требуется найти число обусловленности матрицы рассматриваемой СЛАУ и
относительную погрешность в решении приближённой СЛАУ (3). Затем,
прокомментировать получившиеся результаты. Коме того, найти решение СЛАУ, которая
получается из СЛАУ (3) делением каждого eё $i$-го уравнения $(i=\overline{1,10})$
на число
$b^i + \Delta b^i $. 
После этого сравнить абсолютную погрешность в решении получившейся СЛАУ с
абсолютной погрешностью в решении приближённой СЛАУ (3).

\begin{center}
	\textbf{\textit{Решение}}
\end{center}

\begin{equation*}
	\scriptsize
	A =  \left(
	\begin{array}{cccccccccc}
0,106656 & 0,121277 & 0,133668 & 0,143601 & 0,150893 & 0,155411 & 0,157072 & 0,155843 & 0,15175 & 0,144865 \\
0,121277 & 0,135663 & 0,146566 & 0,153708 & 0,156905 & 0,156074 & 0,151238 & 0,142519 & 0,130143 & 0,114426 \\
0,133668 & 0,146566 & 0,15446 & 0,157079 & 0,154335 & 0,14632 & 0,133308 & 0,115745 & 0,094229 & 0,069495 \\
0,143601 & 0,153708 & 0,157079 & 0,153566 & 0,143323 & 0,126799 & 0,104717 & 0,078047 & 0,047956 & 0,015763 \\
0,150893 & 0,156905 & 0,154335 & 0,143323 & 0,124473 & 0,098815 & 0,067752 & 0,032983 & -0,00359 & -0,03996 \\
0,155411 & 0,156074 & 0,14632 & 0,126799 & 0,098815 & 0,064235 & 0,025368 & -0,01519 & -0,05474 & -0,09063 \\
0,157072 & 0,151238 & 0,133308 & 0,104717 & 0,067752 & 0,025368 & -0,01905 & -0,06194 & -0,09987 & -0,12982 \\
0,155843 & 0,142519 & 0,115745 & 0,078047 & 0,032983 & -0,01519 & -0,06194 & -0,10283 & -0,13402 & -0,15257 \\
0,15175 & 0,130143 & 0,094229 & 0,047956 & -0,00359 & -0,05474 & -0,09987 & -0,13402 & -0,15344 & -0,156 \\
0,144865 & 0,114426 & 0,069495 & 0,015763 & -0,03996 & -0,09063 & -0,12982 & -0,15257 & -0,156 & -0,13967 		
	\end{array}
	\right)
\end{equation*}
\normalsize
Исходная СЛАУ
\begin{equation*}
	\left(E + \uplambda A  \right) {}^{\mathsmaller{>}} x = 
		{}^{\mathsmaller{>}} b
\end{equation*}

\begin{equation*}
	\scriptsize
	E + \uplambda A =
	 \left(
	\begin{array}{cccccccccc}
		1,054394 & 0,061851 & 0,068171 & 0,073236 & 0,076956 & 0,07926 & 0,080106 & 0,07948 & 0,077392 & 0,073881 \\
		0,061851 & 1,069188 & 0,074749 & 0,078391 & 0,080021 & 0,079598 & 0,077131 & 0,072685 & 0,066373 & 0,058357 \\
		0,068171 & 0,074749 & 1,078775 & 0,08011 & 0,078711 & 0,074623 & 0,067987 & 0,05903 & 0,048057 & 0,035442 \\
		0,073236 & 0,078391 & 0,08011 & 1,078319 & 0,073095 & 0,064667 & 0,053406 & 0,039804 & 0,024457 & 0,008039 \\
		0,076956 & 0,080021 & 0,078711 & 0,073095 & 1,063481 & 0,050395 & 0,034553 & 0,016822 & -0,00183 & -0,02038 \\
		0,07926 & 0,079598 & 0,074623 & 0,064667 & 0,050395 & 1,03276 & 0,012938 & -0,00775 & -0,02792 & -0,04622 \\
		0,080106 & 0,077131 & 0,067987 & 0,053406 & 0,034553 & 0,012938 & 0,990287 & -0,03159 & -0,05093 & -0,06621 \\
		0,07948 & 0,072685 & 0,05903 & 0,039804 & 0,016822 & -0,00775 & -0,03159 & 0,947555 & -0,06835 & -0,07781 \\
		0,077392 & 0,066373 & 0,048057 & 0,024457 & -0,00183 & -0,02792 & -0,05093 & -0,06835 & 0,921743 & -0,07956 \\
		0,073881 & 0,058357 & 0,035442 & 0,008039 & -0,02038 & -0,04622 & -0,06621 & -0,07781 & -0,07956 & 0,92877
	\end{array}
	\right)
\end{equation*}

\begin{equation*}
	\scriptsize
	(E + \uplambda A)^{-1} =
	\left(
	\begin{array}{cccccccccc}
0,993122 & -0,01719 & -0,03001 & -0,04451 & -0,05961 & -0,07412 & -0,08684 & -0,09668 & -0,10275 & -0,10448 \\
-0,01719 & 0,97347 & -0,03736 & -0,04894 & -0,06037 & -0,07065 & -0,0788 & -0,08394 & -0,08539 & -0,08275 \\
-0,03001 & -0,03736 & 0,955278 & -0,05157 & -0,05731 & -0,06129 & -0,06295 & -0,06182 & -0,05759 & -0,05018 \\
-0,04451 & -0,04894 & -0,05157 & 0,947814 & -0,05061 & -0,04677 & -0,04067 & -0,03243 & -0,02228 & -0,01058 \\
-0,05961 & -0,06037 & -0,05731 & -0,05061 & 0,959282 & -0,02823 & -0,01391 & 0,001389 & 0,016775 & 0,031368 \\
-0,07412 & -0,07065 & -0,06129 & -0,04677 & -0,02823 & 0,992896 & 0,01498 & 0,036343 & 0,055391 & 0,070731 \\
-0,08684 & -0,0788 & -0,06295 & -0,04067 & -0,01391 & 0,01498 & 1,043454 & 0,06901 & 0,08941 & 0,102878 \\
-0,09668 & -0,08394 & -0,06182 & -0,03243 & 0,001389 & 0,036343 & 0,06901 & 1,096176 & 0,115158 & 0,124062 \\
-0,10275 & -0,08539 & -0,05759 & -0,02228 & 0,016775 & 0,055391 & 0,08941 & 0,115158 & 1,129843 & 0,131857 \\
-0,10448 & -0,08275 & -0,05018 & -0,01058 & 0,031368 & 0,070731 & 0,102878 & 0,124062 & 0,131857 & 1,125441 
	\end{array}
	\right)
\end{equation*}
Следовательно, $||(E + \uplambda A)|| = 1,724729 , \hspace{0.2cm}
 ||(E + \uplambda A)^{-1}||=1,834334$ и число обусловленности
 $cond(E + \uplambda A) = ||(E + \uplambda A)|| \cdot ||(E + \uplambda A)^{-1}|| = 3,16373$.
 Таким образом, матрица СЛАУ (3)
 хорошо обусловлена
\vskip 0.7cm
Решение СЛАУ
$ {}^{\mathsmaller{>}} x = \left(E + \uplambda A  \right) ^{-1} 
{}^{\mathsmaller{>}} b $ , согласно условию 
${}^{\mathsmaller{>}} x = [1,1,\dots,1\rangle \in {}^{\mathsmaller{>}}
\mathbb{R}^{10}
$ \\
Поэтому

\begin{equation*}
	\scriptsize
	{}^{\mathsmaller{>}} b =
	\left(
	\begin{array}{c}
		1,724729 \\ 1,718344 \\ 1,665654 \\ 1,573526 \\ 1,451822 \\ 1,312353 \\ 1,167679 \\ 1,029877 \\ 0,90943 \\ 0,814309
	\end{array}
	\right) ,
	\hspace{0.2cm}
	||{}^{\mathsmaller{>}} b|| = 1,724729
\end{equation*}
Вычислим погрешность решения СЛАУ:
\begin{equation*}
	\scriptsize
	\Delta {}^{\mathsmaller{>}} b =
	\left(
	\begin{array}{c}
		0,017247 \\ -0,01718 \\ 0,016657 \\ -0,01574 \\ 0,014518 \\ -0,01312 \\ 0,011677 \\ -0,0103 \\ 0,009094 \\ -0,00814
	\end{array}
	\right) ,
	\hspace{0.2cm}
	|| \Delta {}^{\mathsmaller{>}} b || = 0,017247 ,
	\hspace{0.2cm}
	 {}^{\mathsmaller{>}} b + \Delta {}^{\mathsmaller{>}} b = 
	 \left(
	 \begin{array}{c}
	 	1,741976 \\ 1,701161 \\ 1,682311 \\ 1,55779 \\ 1,46634 \\ 1,29923 \\ 1,179355 \\ 1,019579 \\ 0,918524 \\ 0,806166
	 \end{array}
	 \right)
\end{equation*}
Найдем решение приближенной СЛАУ:
\begin{equation*}
	\scriptsize
	{}^{\mathsmaller{>}} x + {}^{\mathsmaller{>}} \Delta x = \left(E + \uplambda A  \right) ^{-1} 
	\left( 
	{}^{\mathsmaller{>}} b + 
	{}^{\mathsmaller{>}} \Delta b
	\right)
\end{equation*}
\begin{equation*}
	\scriptsize
	{}^{\mathsmaller{>}} x + {}^{\mathsmaller{>}} \Delta x = 
		\left(
	\begin{array}{cccccccccc}
		0,993122 & -0,01719 & -0,03001 & -0,04451 & -0,05961 & -0,07412 & -0,08684 & -0,09668 & -0,10275 & -0,10448 \\
		-0,01719 & 0,97347 & -0,03736 & -0,04894 & -0,06037 & -0,07065 & -0,0788 & -0,08394 & -0,08539 & -0,08275 \\
		-0,03001 & -0,03736 & 0,955278 & -0,05157 & -0,05731 & -0,06129 & -0,06295 & -0,06182 & -0,05759 & -0,05018 \\
		-0,04451 & -0,04894 & -0,05157 & 0,947814 & -0,05061 & -0,04677 & -0,04067 & -0,03243 & -0,02228 & -0,01058 \\
		-0,05961 & -0,06037 & -0,05731 & -0,05061 & 0,959282 & -0,02823 & -0,01391 & 0,001389 & 0,016775 & 0,031368 \\
		-0,07412 & -0,07065 & -0,06129 & -0,04677 & -0,02823 & 0,992896 & 0,01498 & 0,036343 & 0,055391 & 0,070731 \\
		-0,08684 & -0,0788 & -0,06295 & -0,04067 & -0,01391 & 0,01498 & 1,043454 & 0,06901 & 0,08941 & 0,102878 \\
		-0,09668 & -0,08394 & -0,06182 & -0,03243 & 0,001389 & 0,036343 & 0,06901 & 1,096176 & 0,115158 & 0,124062 \\
		-0,10275 & -0,08539 & -0,05759 & -0,02228 & 0,016775 & 0,055391 & 0,08941 & 0,115158 & 1,129843 & 0,131857 \\
		-0,10448 & -0,08275 & -0,05018 & -0,01058 & 0,031368 & 0,070731 & 0,102878 & 0,124062 & 0,131857 & 1,125441 
	\end{array}
	\right)
	\cdot
\end{equation*}
\begin{equation*}
	\scriptsize
	\cdot
	\left(
	\begin{array}{c}
		1,741976 \\ 1,701161 \\ 1,682311 \\ 1,55779 \\ 1,46634 \\ 1,29923 \\ 1,179355 \\ 1,019579 \\ 0,918524 \\ 0,806166
	\end{array}
	\right) = 
	\left(
	\begin{array}{c}
		1,01763 \\ 0,983016 \\ 1,016606 \\ 0,983922 \\ 1,013869 \\ 0,985939 \\ 1,010498 \\ 0,988352 \\ 1,007663 \\ 0,990436
	\end{array}
	\right) , \hspace{0.3cm}
	{}^{\mathsmaller{>}} \Delta x = 
	\left(
	\begin{array}{c}
		0,01763 \\ -0,01698 \\ 0,016606 \\ -0,01608 \\ 0,013869 \\ -0,01406 \\ 0,010498 \\ -0,01165 \\ 0,007663 \\ -0,00956
	\end{array}
	\right)
\end{equation*}
Поскольку $|| {}^{\mathsmaller{>}} x ""|| = 1 $
и 
$|| \Delta 	{}^{\mathsmaller{>}} x ""|| = 0,01763 $
относительная ошибка вычислений
$ \dfrac{|| \Delta 	{}^{\mathsmaller{>}} x ""||}{|| {}^{\mathsmaller{>}} x ""||}  =
0,01763 \leqslant cond(E + \uplambda A)
\dfrac{|| \Delta {}^{\mathsmaller{>}} b ""||}{|| {}^{\mathsmaller{>}} b ""||} =
3,16373 \cdot 0,01 = 0,031637	$ \\
Следствием хорошей обусловленности является малая погрешность.

\textbf{Найдем решение СЛАУ, которая получается из СЛАУ (3) делением каждого её
$i$-го уравнения  $(i=\overline{1,10})$
на число
$b^i + \Delta b^i $. } \\
Получим матрицу 
\begin{equation*}
\scriptsize
B =  \left(
\begin{array}{cccccccccc}
	0,605287 & 0,035506 & 0,039134 & 0,042042 & 0,044177 & 0,0455 & 0,045986 & 0,045626 & 0,044428 & 0,042412 \\
	0,036358 & 0,628505 & 0,04394 & 0,046081 & 0,047039 & 0,04679 & 0,04534 & 0,042727 & 0,039016 & 0,034304 \\
	0,040522 & 0,044432 & 0,641246 & 0,047619 & 0,046787 & 0,044357 & 0,040413 & 0,035089 & 0,028566 & 0,021068 \\
	0,047013 & 0,050322 & 0,051426 & 0,69221 & 0,046922 & 0,041512 & 0,034283 & 0,025552 & 0,0157 & 0,005161 \\
	0,052481 & 0,054572 & 0,053678 & 0,049848 & 0,725262 & 0,034368 & 0,023564 & 0,011472 & -0,00125 & -0,0139 \\
	0,061005 & 0,061265 & 0,057436 & 0,049774 & 0,038789 & 0,794902 & 0,009958 & -0,00596 & -0,02149 & -0,03558 \\
	0,067924 & 0,065401 & 0,057648 & 0,045284 & 0,029299 & 0,01097 & 0,839685 & -0,02678 & -0,04319 & -0,05614 \\
	0,077954 & 0,071289 & 0,057896 & 0,03904 & 0,016499 & -0,0076 & -0,03098 & 0,92936 & -0,06704 & -0,07632 \\
	0,084257 & 0,07226 & 0,052319 & 0,026627 & -0,00199 & -0,03039 & -0,05545 & -0,07442 & 1,003504 & -0,08661 \\
	0,091645 & 0,072388 & 0,043964 & 0,009972 & -0,02528 & -0,05734 & -0,08213 & -0,09652 & -0,09869 & 1,152082		
\end{array}
\right)
\end{equation*}

\begin{equation*}
	\scriptsize
	B^{-1} =  \left(
	\begin{array}{cccccccccc}
		1,729994 & -0,02925 & -0,05049 & -0,06933 & -0,0874 & -0,0963 & -0,10242 & -0,09857 & -0,09438 & -0,08423 \\
		-0,02995 & 1,65603 & -0,06285 & -0,07624 & -0,08853 & -0,0918 & -0,09293 & -0,08558 & -0,07843 & -0,06671 \\
		-0,05228 & -0,06355 & 1,607075 & -0,08034 & -0,08403 & -0,07963 & -0,07424 & -0,06303 & -0,0529 & -0,04045 \\
		-0,07753 & -0,08326 & -0,08676 & 1,476495 & -0,07422 & -0,06077 & -0,04797 & -0,03307 & -0,02047 & -0,00853 \\
		-0,10383 & -0,1027 & -0,09641 & -0,07885 & 1,406634 & -0,03668 & -0,01641 & 0,001416 & 0,015408 & 0,025288 \\
		-0,12911 & -0,12019 & -0,10312 & -0,07286 & -0,04139 & 1,290001 & 0,017667 & 0,037055 & 0,050878 & 0,057021 \\
		-0,15128 & -0,13405 & -0,10591 & -0,06336 & -0,0204 & 0,019462 & 1,230603 & 0,070361 & 0,082126 & 0,082937 \\
		-0,16841 & -0,14279 & -0,104 & -0,05052 & 0,002036 & 0,047219 & 0,081387 & 1,117638 & 0,105775 & 0,100015 \\
		-0,17899 & -0,14526 & -0,09689 & -0,03471 & 0,024597 & 0,071966 & 0,105447 & 0,117413 & 1,037788 & 0,106299 \\
		-0,182 & -0,14078 & -0,08442 & -0,01649 & 0,045997 & 0,091895 & 0,12133 & 0,126491 & 0,121114 & 0,907292		
	\end{array}
	\right)
\end{equation*}
\begin{equation*}
	\scriptsize
{}^{\mathsmaller{>}} x  ^\prime = B^{-1} \cdot
{}^{\mathsmaller{>}} x  ^\prime = 
\left(
\begin{array}{c}
	1,01763 \\ 0,983016 \\ 1,016606 \\ 0,983922 \\ 1,013869 \\ 0,985939 \\ 1,010498 \\ 0,988352 \\ 1,007663 \\ 0,990436
\end{array}
\right) , \hspace{0.3cm}
{}^{\mathsmaller{>}} \Delta x  ^\prime = 
\left(
\begin{array}{c}
	0,01763 \\ -0,01698 \\ 0,016606 \\ -0,01608 \\ 0,013869 \\ -0,01406 \\ 0,010498 \\ -0,01165 \\ 0,007663 \\ -0,00956
\end{array}
\right)  , \hspace{0.3cm}
|| \Delta 	{}^{\mathsmaller{>}} x^\prime|| = 0,01763 
\end{equation*}

Таким образом, абсолютная погрешности
$ || {}^{\mathsmaller{>}} \Delta x  ^\prime ||  = 0,01763 $ и 
$  || {}^{\mathsmaller{>}} \Delta x  ||  = 0,01763 
$ решений
приближенной и исходной СЛАУ совпадают.

\begin{center}
	\textbf{\textit{Результаты}}
\end{center}
Число обусловленности матрицы СЛАУ (3)
$ cond(E+ \uplambda A) = 3,16373 < 10 $ , то есть матрица
$ E+ \uplambda A $ хорошо обусловлена. Следствием этого является малая относительная погрешность
при решении приближенной СЛАУ:
$ \dfrac{|| \Delta {}^{\mathsmaller{>}} x ||}{|| {}^{\mathsmaller{>}} x ""||} =
 0,01763
$ .
Эта погрешность в 1,7945 раз меньше
верхней границы относительной погрешности
$cond(E+ \uplambda A)
 \dfrac{|| \Delta {}^{\mathsmaller{>}} b ||}{|| {}^{\mathsmaller{>}} b ""||} =
 0,031637
$ .
Кроме
того, при делении каждого $i$-го уравнения  $(i=\overline{1,10})$
СЛАУ на число $b^i + \Delta b^i $ абсолютная
погрешность не изменилась:
$ || \Delta 	{}^{\mathsmaller{>}} x^\prime|| = 
|| \Delta 	{}^{\mathsmaller{>}} x|| =
0,01763 $.
\end{document}